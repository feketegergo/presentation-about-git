%% LaTeX Beamer presentation template (requires beamer package)
%% see http://bitbucket.org/rivanvx/beamer/wiki/Home
%% idea contributed by H. Turgut Uyar
%% template based on a template by Till Tantau
%% this template is still evolving - it might differ in future releases!

\documentclass{beamer}

\mode<presentation>
{
\usetheme{Warsaw}

\setbeamercovered{transparent}
}

\usepackage[english]{babel}
\usepackage[latin1]{inputenc}

% font definitions, try \usepackage{ae} instead of the following
% three lines if you don't like this look
\usepackage{mathptmx}
\usepackage[scaled=.90]{helvet}
\usepackage{courier}


\usepackage[T1]{fontenc}


\title{Colony Size Detection  -  Overexpression}

%\subtitle{}

% - Use the \inst{?} command only if the authors have different
%   affiliation.
%\author{F.~Author\inst{1} \and S.~Another\inst{2}}
\author{Gergely Fekete}

% - Use the \inst command only if there are several affiliations.
% - Keep it simple, no one is interested in your street address.
%\institute[Universities of]
%{
%\inst{1}%
%Department of Computer Science\\
%Univ of S
%\and
%\inst{2}%
%Department of Theoretical Philosophy\\
%Univ of E}

\date{2014.08.12 -  Lab Meeting}


% This is only inserted into the PDF information catalog. Can be left
% out.
\subject{Overexpression - Threated Strains}



% If you have a file called "university-logo-filename.xxx", where xxx
% is a graphic format that can be processed by latex or pdflatex,
% resp., then you can add a logo as follows:

% \pgfdeclareimage[height=0.5cm]{university-logo}{university-logo-filename}
% \logo{\pgfuseimage{university-logo}}




% If you wish to uncover everything in a step-wise fashion, uncomment
% the following command:

%\beamerdefaultoverlayspecification{<+->}

\begin{document}

\begin{frame}
\titlepage
\end{frame}

\section{gitter}
\subsection{The Experiment}



\begin{frame}
\frametitle<presentation>{The Experiment}
\includegraphics[height=60pt]{figures/orf-in-plasmid.jpg}
\begin{itemize}
   \item  We have sequencing data
   \item 1 in data on $t_0$ time
   \item 4 in data on $t_1$ time: 3 threatments and a non threated
   \item We calculate 4 fitnes values (statisticaly dependent)
   
   
   \item no background levels measured
   \item no replicates
   
   \item \ldots but p-values are needed
   
 \end{itemize}
\end{frame}



\begin{frame}
\frametitle<presentation>{Treshold}
\begin{itemize}
  
  
  
   \item $t_0$ < tershold  AND all of $t_1$< treshold $\Rightarrow$ drop row
 \end{itemize}
\end{frame}




\end{document}
